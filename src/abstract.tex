本論文は音楽のためのプログラミング言語mimiumの実装を通じて、表現メディウム、またインフラストラクチャとしてのプログラミング言語設計の理論的枠組みの構築を行うものである。

80年代以降のメディア論、また90年代以降のインフラストラクチャ研究にはパーソナルコンピューターの普及を背景にして、その理論的基盤の中に、電子計算機そのものの理論的基盤でもある、あらゆる情報を電気的な信号として伝達する理論、サイバネティクスを取り込んできた。
一方コンピューターを万能メディア装置として使う思想の背景にはマーシャル・マクルーハンのようなメディア研究の思想が色濃く反映されてもいる。

音楽のためのプログラミング言語は単に音楽を作るためのツールではない。コンピューターを使って音楽を作るための入力、保存のための形式として黎明期には必然的に選ばれていたテキストというデータ構造が、GUIやタンジブルなインターフェースが成立した後にも、けもの道のように残った痕跡の上に並行して発展してきた汎用プログラミング言語の理論が流入することで使われ続け、今も発展を続けている。その中でライブコーディングのような対話的インターフェースとしての側面や、1つのコードを様々なプラットフォームで動作させるためのインフラストラクチャ/プロトコルとしての役割という異なる性質を帯びるようになってきた。

mimiumはそうした歴史的経緯や慣習的なものを一旦
